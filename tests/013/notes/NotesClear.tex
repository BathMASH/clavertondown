% Options for packages loaded elsewhere
\PassOptionsToPackage{unicode}{hyperref}
\PassOptionsToPackage{hyphens}{url}
%
\documentclass[
  12pt,
  a4paper]{extarticle}
\usepackage{amsmath,amssymb}
\usepackage{lmodern}
\usepackage{iftex}
\ifPDFTeX
  \usepackage[T1]{fontenc}
  \usepackage[utf8]{inputenc}
  \usepackage{textcomp} % provide euro and other symbols
\else % if luatex or xetex
  \usepackage{unicode-math}
  \defaultfontfeatures{Scale=MatchLowercase}
  \defaultfontfeatures[\rmfamily]{Ligatures=TeX,Scale=1}
\fi
% Use upquote if available, for straight quotes in verbatim environments
\IfFileExists{upquote.sty}{\usepackage{upquote}}{}
\IfFileExists{microtype.sty}{% use microtype if available
  \usepackage[]{microtype}
  \UseMicrotypeSet[protrusion]{basicmath} % disable protrusion for tt fonts
}{}
\makeatletter
\@ifundefined{KOMAClassName}{% if non-KOMA class
  \IfFileExists{parskip.sty}{%
    \usepackage{parskip}
  }{% else
    \setlength{\parindent}{0pt}
    \setlength{\parskip}{6pt plus 2pt minus 1pt}}
}{% if KOMA class
  \KOMAoptions{parskip=half}}
\makeatother
\usepackage{xcolor}
\IfFileExists{xurl.sty}{\usepackage{xurl}}{} % add URL line breaks if available
\IfFileExists{bookmark.sty}{\usepackage{bookmark}}{\usepackage{hyperref}}
\hypersetup{
  pdftitle={Test 013: New theorems, numbering and styles with lists immediately inside and a newline present in clear and large},
  pdfauthor={Emma Cliffe, Skills Centre: MASH, University of Bath},
  hidelinks,
  pdfcreator={LaTeX via pandoc}}
\urlstyle{same} % disable monospaced font for URLs
\usepackage[margin=2.5cm]{geometry}
\usepackage{longtable,booktabs,array}
\usepackage{calc} % for calculating minipage widths
% Correct order of tables after \paragraph or \subparagraph
\usepackage{etoolbox}
\makeatletter
\patchcmd\longtable{\par}{\if@noskipsec\mbox{}\fi\par}{}{}
\makeatother
% Allow footnotes in longtable head/foot
\IfFileExists{footnotehyper.sty}{\usepackage{footnotehyper}}{\usepackage{footnote}}
\makesavenoteenv{longtable}
\usepackage{graphicx}
\makeatletter
\def\maxwidth{\ifdim\Gin@nat@width>\linewidth\linewidth\else\Gin@nat@width\fi}
\def\maxheight{\ifdim\Gin@nat@height>\textheight\textheight\else\Gin@nat@height\fi}
\makeatother
% Scale images if necessary, so that they will not overflow the page
% margins by default, and it is still possible to overwrite the defaults
% using explicit options in \includegraphics[width, height, ...]{}
\setkeys{Gin}{width=\maxwidth,height=\maxheight,keepaspectratio}
% Set default figure placement to htbp
\makeatletter
\def\fps@figure{htbp}
\makeatother
\setlength{\emergencystretch}{3em} % prevent overfull lines
\providecommand{\tightlist}{%
  \setlength{\itemsep}{0pt}\setlength{\parskip}{0pt}}
\setcounter{secnumdepth}{5}
\usepackage{needspace}
\ifLuaTeX
  \usepackage{selnolig}  % disable illegal ligatures
\fi

\title{Test 013: New theorems, numbering and styles with lists immediately inside and a newline present in clear and large}
\author{Emma Cliffe, Skills Centre: MASH, University of Bath}
\date{September 2022}

\usepackage{amsthm}
\theoremstyle{plain}
\newtheorem*{theorem*}{Theorem}\newtheorem{theorem}{Theorem}[section]
\newtheorem*{Thought*}{Thought}\newtheorem{Thought}{Thought}[section]
\theoremstyle{definition}
\newtheorem*{proposition*}{Proposition}\newtheorem{proposition}[theorem]{Proposition}
\newtheorem*{Nugget*}{Nugget}\newtheorem{Nugget}[Thought]{Nugget}
\theoremstyle{plain}
\newtheorem*{lemma*}{Lemma}\newtheorem{lemma}{Lemma}[section]
\theoremstyle{plain}
\newtheorem*{corollary*}{Corollary}\newtheorem{corollary}{Corollary}[section]
\theoremstyle{plain}
\newtheorem*{conjecture*}{Conjecture}\newtheorem{conjecture}{Conjecture}[section]
\theoremstyle{definition}
\newtheorem*{definition*}{Definition}\newtheorem{definition}{Definition}[section]
\theoremstyle{definition}
\newtheorem*{example*}{Example}\newtheorem{example}{Example}[section]
\theoremstyle{definition}
\newtheorem*{exercise*}{Exercise}\newtheorem{exercise}{Exercise}[section]
\newtheorem*{Solution*}{Solution}\newtheorem{Solution}{Solution}[section]
\theoremstyle{remark}
\newtheorem*{remark*}{Remark}
\newtheorem*{solution*}{Solution}
\theoremstyle{plain}
\newtheorem*{Defns*}{Defns}
\newtheorem*{Proof*}{Proof}
\newtheorem*{Exercises*}{Exercises}
\newtheorem*{Example*}{Example}
\let\BeginKnitrBlock\begin \let\EndKnitrBlock\end


%\usepackage[english,shorthands=off]{babel}
\usepackage{etoolbox}
\usepackage{spverbatim}
\makeatletter
\@ifpackageloaded{float}{}{\usepackage{float}}
\@ifpackageloaded{adjustbox}{}{\usepackage[Export]{adjustbox}}
\makeatother
\floatplacement{figure}{H}
\newcommand{\scalefactor}{1.2}
\adjustboxset*{min width=\scalefactor\width,max width=\linewidth}
\renewcommand{\familydefault}{phv}
\fontfamily{phv}\selectfont
\renewcommand{\em}{\bf}\renewcommand{\textit}{\textbf}\renewcommand{\emph}{\textbf}\renewcommand{\it}{\bf}\renewcommand{\itshape}{\bf}
\setlength{\parindent}{0.0pt}
\setlength{\parskip}{1.0\baselineskip}
\renewcommand{\baselinestretch}{1.5}\selectfont
\setlength{\mathsurround}{0.2em}
\setlength{\arraycolsep}{0.5cm}\renewcommand{\arraystretch}{1.5}
\addtolength{\jot}{\baselineskip}
\renewcommand{\;}{\,}
\sloppy
\allowdisplaybreaks
\usepackage{amsthm}
\newtheoremstyle{plain}{20pt}{3pt}{}{}{\bfseries}{.\newline\nobreak}{1.0em\nobreak}{}
\newtheoremstyle{definition}{20pt}{3pt}{}{}{\bfseries}{.\newline\nobreak}{1.0em\nobreak}{}
\newtheoremstyle{remark}{20pt}{3pt}{}{}{\bfseries}{.\newline\nobreak}{1.0em\nobreak}{}
\csundef{Proof}
\csundef{endProof}
\newenvironment{Proof}
  {\noindent{\bf Proof.}\hspace*{1em}}% Begin
  {\qed\par}% End
%% When redefining an environment it is vital that it has 
%% the same number of arguments as the original
\renewenvironment{proof}[1][\proofname]
  {\trivlist\item\relax\noindent{\bf {#1}.}\hspace*{1em}}% Begin
  {\qed\endtrivlist}% End

\begin{document}
\maketitle

{
\setcounter{tocdepth}{2}
\tableofcontents
}
\hypertarget{itemize-without-theorem}{%
\section{Itemize without theorem}\label{itemize-without-theorem}}

\begin{itemize}
\tightlist
\item
  One
\item
  Two
\end{itemize}

\hypertarget{testing-some-theorem-stuff}{%
\section{Testing some theorem stuff}\label{testing-some-theorem-stuff}}

\BeginKnitrBlock{definition}
{\label{def:truth} }
\leavevmode\vspace{-\baselinestretch\parskip}\nobreak

\begin{itemize}
\tightlist
\item
  One
\item
  Two
\end{itemize}

Here is a definition.
\EndKnitrBlock{definition}

\BeginKnitrBlock{example}
{\label{exm:unnamed-chunk-1} }
\leavevmode\vspace{-\baselinestretch\parskip}\nobreak

\begin{itemize}
\tightlist
\item
  One
\item
  Two
\end{itemize}

Here is an example.
\EndKnitrBlock{example}

Here is some more boring text in between.

\BeginKnitrBlock{theorem}[Foo]
{\label{thm:thm1} }
\leavevmode\vspace{-\baselinestretch\parskip}\nobreak

\begin{itemize}
\tightlist
\item
  One
\item
  Two
\end{itemize}

Bookdown is needed for things like theorems and internal references
\EndKnitrBlock{theorem}

\BeginKnitrBlock{proposition}[Thingy we need for \ref{thm:thm1}]
{\label{prp:prp1} }
\leavevmode\vspace{-\baselinestretch\parskip}\nobreak

\begin{itemize}
\tightlist
\item
  One
\item
  Two
\end{itemize}

You can create new theorem types
\EndKnitrBlock{proposition}

\needspace{\baselineskip}

\BeginKnitrBlock{Thought}[Bar of \ref{thm:thm1}]
{\label{tho:tho1}}
\leavevmode\vspace{-\baselinestretch\parskip}\nobreak

\begin{itemize}
\tightlist
\item
  One
\item
  Two
\end{itemize}

You can create new theorem types
\EndKnitrBlock{Thought}

\BeginKnitrBlock{Proof*}[Of theorem \ref{thm:thm1}]
{}
\leavevmode\vspace{-\baselinestretch\parskip}\nobreak

\begin{itemize}
\tightlist
\item
  One
\item
  Two
\end{itemize}

Here is a proof\qed
\EndKnitrBlock{Proof*}

\BeginKnitrBlock{proof}[Proof of theorem \ref{thm:thm1}]
\leavevmode\vspace{-\baselinestretch\parskip}\nobreak

\begin{itemize}
\tightlist
\item
  One
\item
  Two
\end{itemize}

Here is a proof
\EndKnitrBlock{proof}

\BeginKnitrBlock{Defns*}
{}
\leavevmode\vspace{-\baselinestretch\parskip}\nobreak

\begin{itemize}
\tightlist
\item
  One
\item
  Two
\end{itemize}

You can create new unumbered theorem types
\EndKnitrBlock{Defns*}

\BeginKnitrBlock{Nugget}
{\label{nug:nug1}}
\leavevmode\vspace{-\baselinestretch\parskip}\nobreak

\begin{itemize}
\tightlist
\item
  One
\item
  Two
\end{itemize}

You can create new theorem types
\EndKnitrBlock{Nugget}

\BeginKnitrBlock{Example*}
{}
\leavevmode\vspace{-\baselinestretch\parskip}\nobreak

\begin{itemize}
\tightlist
\item
  One
\item
  Two
\end{itemize}

An example
\EndKnitrBlock{Example*}

\BeginKnitrBlock{Solution}
{\label{sol:sol1}}
\leavevmode\vspace{-\baselinestretch\parskip}\nobreak

\begin{itemize}
\tightlist
\item
  One
\item
  Two
\end{itemize}

You can create new theorem types
\EndKnitrBlock{Solution}

\BeginKnitrBlock{Exercises*}
{}
\leavevmode\vspace{-\baselinestretch\parskip}\nobreak

\begin{itemize}
\tightlist
\item
  One
\item
  Two
\end{itemize}

Here is a question
\EndKnitrBlock{Exercises*}

\BeginKnitrBlock{solution*}
\leavevmode\vspace{-\baselinestretch\parskip}\nobreak

\begin{itemize}
\tightlist
\item
  One
\item
  Two
\end{itemize}

Test
\EndKnitrBlock{solution*}

\hypertarget{testing-the-reference-link-back}{%
\section{Testing the reference link back}\label{testing-the-reference-link-back}}

Now go to theorem \ref{thm:thm1} or thought \ref{tho:tho1}

\BeginKnitrBlock{solution*}
\leavevmode\vspace{-\baselinestretch\parskip}\nobreak

\begin{itemize}
\tightlist
\item
  One
\item
  Two
\end{itemize}

Test
\EndKnitrBlock{solution*}

\needspace{3\baselineskip}
\begingroup\renewcommand{\thedefinition}{\ref{def:truth}}
\BeginKnitrBlock{definition}
{ }
\leavevmode\vspace{-\baselinestretch\parskip}\nobreak

\begin{itemize}
\tightlist
\item
  One
\item
  Two
\end{itemize}

Here is a definition.
\EndKnitrBlock{definition}
\endgroup\addtocounter{definition}{-1}\begingroup\renewcommand{\theThought}{\ref{tho:tho1}}
\BeginKnitrBlock{Thought}[Bar of \ref{thm:thm1}]
{}
\leavevmode\vspace{-\baselinestretch\parskip}\nobreak

\begin{itemize}
\tightlist
\item
  One
\item
  Two
\end{itemize}

You can create new theorem types
\EndKnitrBlock{Thought}
\endgroup\addtocounter{Thought}{-1}
Now an actual new thing:
\BeginKnitrBlock{Thought}
{\label{tho:tho2}}
\leavevmode\vspace{-\baselinestretch\parskip}\nobreak

\begin{itemize}
\tightlist
\item
  One
\item
  Two
\end{itemize}

Stuff and nonsense
\EndKnitrBlock{Thought}

\end{document}
